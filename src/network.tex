\part{Network}

\chapter{Overview}


计算机网络是硬件和软件的组合,其中硬件把信号从网络中的一个节点传送到另一个节点,而软件用于使用户可以使用网络服务。

网络的性能可以用多种方式来衡量,例如传输时间、响应时间等。

\begin{compactitem}
\item 传输时间是信息从一个设备传输到另一个设备所需的时间总量。
\item 响应时间是查询和响应之间的时间间隔。
\end{compactitem}

网络的性能依赖于许多因素,例如用户数、传输介质类型、硬件的连接能力和软件的效率等。

除了发送的准确性之外,还可以从发生故障的频率、从故障中恢复的时间、灾难时网络的健壮性等因素来衡量网络的可靠性。

网络的安全问题包括数据保护、防范非授权访问、破坏和修改,实现从数据破坏和数据丢失中恢复的策略和程序等。

\section{Connection}

链路是数据从一个设备传输到另一个设备的通信信道,可能的连接类型包括点对点和多点。

\begin{compactitem}
\item 点对点连接提供了两个设备间的专用链路,而且链路的整个容量为两个设备的传输所拥有。
\item 多点连接(或多站连接)是两个以上的指定设备共享一个链路,因此信道的容量被共享。
\end{compactitem}

\section{Topology}

网络的拓扑是所有链路和设备(通常为节点)间关系的几何表示,包括网状型、星型、总线型和环型。

实际上,网络的物理拓扑都是两个或多个设备连接到一个链路,一个或多个链路形成拓扑。

\begin{compactitem}
\item 在网状拓扑中,每个设备都有专用的点对点链路与其他每个设备相连。
\item 在星型拓扑中,每个设备都有专用的点对点链路,只与集线器(Hub)相连。
\item 在总线拓扑中,所有的设备都连接到总线,而且每个节点使用分支线和连接器与总线相连。
\item 在环形拓扑中,每个设备都有专用的点对点链路,并且只与两边的设备相连,信号只是以一个方向沿着环从一个设备传输到另一个设备,直到到达目的地。
\end{compactitem}•

在环形拓扑中,环中的每个设备连接一个中继器(Router),这样当一个设备收到要发送到另一个设备的信号时,中继器可以重新生成二进制位并传输它们。

星型拓扑比网状拓扑更便宜,而且具有网状拓扑的大多数优点,只是其整个拓扑依赖单个点(即集线器)。

\section{Router}



当两个或多个网络通过路由器连接在一起时,它们就变成了互联网(internet),其中最著名的互联网就是因特网(Internet)。

路由器(Router)是发送数据包(消息),并使其在互联网中传输的连接设备。

\begin{compactitem}
\item 网络本身可以是一组连接在一起的通信设备(例如计算机和打印机)。
\item 互联网本身是能够通过路由器进行互相通信的两个或多个网络。
\item 路由器只使用TCP/IP协议族的前三层(不需要传输层)。
\end{compactitem}

路由器选择协议(例如RIP、OSPF和BGP)向Internet上的所有路由器发送自己的信息,更新它们关于路由的信息,因此路由选择协议可以更新路由器中的路由表。

\chapter{Protocol}

下面的示意图说明了TCP/IP协议族中每层的职责和每层涉及的地址。

\begin{table}[htbp]
\centering
\begin{tabular}{l|c|l}
\hline
\multirow{2}{100pt}{消息} & \fbox{进程} &  \multirow{2}{100pt}{应用层地址}\\
						& 应用层 &\\
\hline
\multirow{2}{100pt}{段、用户数据报或包} & \fbox{SCTP}~\fbox{TCP}~\fbox{UDP} & \multirow{2}{100pt}{端口号}\\
						& 传输层 & \\
\hline
\multirow{2}{100pt}{数据报} & \fbox{IP和其他协议} & \multirow{2}{100pt}{IP地址}\\
						& 网络层 & \\
\hline
\multirow{2}{100pt}{帧} & \fbox{LAN、WAN和MAN协议} & \multirow{2}{100pt}{链路地址}\\
						& 数据链路层& \\
\hline
\multirow{2}{100pt}{位} & \fbox{把位转换为信号} & \\
						& 物理层 & \\
\hline
\end{tabular}
\end{table}

\begin{compactitem}
\item 在应用层,进程交换信息;
\item 在传输层,数据单元被称为段(TCP)、用户数据报(UDP)或包(SCTP);
\item 在网络层,数据单元被称为数据报;
\item 在数据链路层,数据单元被称为帧;
\item 在物理层,数据单元是二进制位。
\end{compactitem}

不同的分层都对数据进行了封装,越具体的层都会将头(或有可能是尾)加到数据单元中。

通常情况下,尾只在第二层被加入,并且在格式化的数据单元经过物理层时被转换为电磁信号,然后沿着物理电路进行传输。

当数据到达目标计算机后就会反向经过各层,当每个数据块到达下一个更高层时,在相应发送层附加上去的头和尾都被移除,并且执行当前层相应的动作,到达最高层时将会被转换为适合应用的形式(即对接收者可用)。

在层中数据单元之间的关系不是一对一的,例如在TCP中将会把消息分解为几个部分,每个部分封装在一个段中,但是UDP中的每个消息将被封装在一个用户数据报中,因此发送给UDP的消息必须足够小,才能保存在一个数据报中。

当段或用户数据报用数据报封装时,情况与上述相同,而且网络层中的数据报也可以分解为数据链路层中的多个帧。






\section{TCP}


使用计算机解决问题的基础工作是由硬件完成的,用户通过软件可以控制问题求解过程,硬件工作的细节问题由软件层处理。

计算机网络提供的服务也可以类比于解决问题,例如发送电子邮件的任务可以被分解为更小的子任务。

实际上,计算机网络的每层完成一个任务,而且每层都使用更低层的服务。

在计算机网络的最底层,信号或信号组被从源计算机传送到目的计算机,这也是网络协议的最底层。

计算机网络协议允许使用不同技术的LAN和WAN互相连接到一起,从而可以从一点向另一点传送信息。

原始的TCP/IP协议族被定义为4层,分别是主机到网络层(或链路层)、互联网层(网络层)、传输层和应用层,现在已经发展为5层。

\begin{table}[htbp]
\centering
\caption{TCP/IP协议族}
\begin{tabular}{|l|l|l|l|l|}
\hline
	& 应用层 & 5 & 应用层 & \\ \cline{2-4}
	&传输层 & 4 & 传输层 & \\ \cline{2-4}
TCP/IP&网络层 & 3 & 互联网层 & 原始的TCP/IP\\ \cline{2-4}
	&数据链路层 & 2 & 主机到网络层 & \\ \cline{2-4}
	&物理层 & 1 & & \\
\hline
\end{tabular}
\end{table}

在网络协议分层结构中,每一层调用其直接下层的服务,路由器只使用TCP/IP协议族的前三层。

应用层允许用户(人或者软件)访问网络,并提供对电子邮件、远程文件访问和传输、浏览Internet等服务的支持。

应用层负责向用户提供服务,而且应用层是唯一一个大多数Internet用户能够直观感受的层。例如,客户/服务器体系结构和对等体系结构都可以用于允许不同计算机上的应用程序互相通信,它们都需要应用层的支持。

在客户/服务器体系结构中,每个应用由两个分开但相关的程序(客户端程序和服务器端程序)组成。

\begin{compactitem}
\item 客户端程序只在需要时运行。
\item 服务器端程序需要一直运行。
\end{compactitem}

客户端程序和服务器端程序之间的通信称为进程到进程的通信,其中服务器端进程一直运行,并等待接收客户端进程的请求。

用户运行休眠的客户端程序就可以将其转变为客户端进程,使得客户端进程请求服务,而且将被服务器端进程响应。

一般情况下,服务器端进程可以响应多个客户端进程的请求并返回数据。

网络层负责源到目的地的数据包发送,实际上可能需要跨越多个网络(链路),并且网络层保证每个数据包从源点到最终目的地。

除了多路复用和解多路复用以外,期望从网络层得到的服务和传输层服务相似,但是实际上在Internet中实现拥塞控制、流量控制和差错控制并不现实。



\section{IP}

在TCP/IP协议族中,网络层的主协议IP(因特网协议)用于从源计算机到目的计算机的数据包发送。



IP地址使用点分十进制记法表示的32位的地址标识,32位的地址被分解为4个8位的部分,每个部分写成0~255的十进制数,这样IPv4(32位)的地址范围可以定义$2^{32}$(超过40亿)个不同的设备。

在消息的源头,IPv4协议把源和目的的IP地址加到从应用层发送过来的数据包中,然后将数据包放入缓冲区准备发送,实际的传输是由数据链路层和物理层来实现的。

从客户端到服务器端的数据包和从服务器端到客户端的数据包都需要网络层地址,而且网络层使用其路由表来找到下一跳(路由器)的逻辑地址,然后把这个地址传递给数据链路层。

\begin{compactitem}
\item 服务器的地址由服务器提供;
\item 客户端地址是客户端计算机提供的。
\end{compactitem}

当客户需要向服务器发送请求时,首先需要服务器应用层的地址,而且服务器应用层地址不能用来发送消息的,它只是帮助客户找到服务器的实际地址,数据链路层使用逻辑地址来找到下一个路由器的数据链路层地址。


在现实生活中,邮局并不保证用户的信件一定被收件人收到,因此IP协议提供的尽力而为服务,不保证数据包准确无误到达或顺序是发送者希望的,也不保证任何数据包都被发送,数据包有可能丢失。如果需要确保数据准确无误地发送到目的计算机,那么就需要使用可靠传输协议(例如TCP)或用户本身实现差错控制来补充IP提供的服务。

实际上,IP使用其他的协议来在一定程度上弥补其不足,例如:

\begin{compactitem}
\item ICMP(Internet控制消息协议)可以用来报告一定数目的差错给源计算机。
\item IGMP(Internet小组管理协议)可以用来增加IP的多播能力。
\item ARP(地址解析协议)和RARP(反向地址解析协议)
\end{compactitem}

在网络拥堵时,路由器丢失数据包后可以由ICMP发送一个数据包给源计算机来警告其拥堵,而且ICMP还可以用来检查Internet节点的状态。

从本质上来说,IP是单播传输的协议,一个源,一个目的地,而多播传输则是一个源,多个目的地,因此通过IGMP来增强IP的多播能力。

在应用层,客户端位置不需要进行标识,不同的服务器有不同的应用层地址。举例来说,在只知道一个人的姓名时无法向其发送信件,邮局无法仅凭姓名投递信件,还需要收信人的实际地址。

应用层地址可以帮助客户端通过路由算法来找到服务器的实际地址(也就是服务器的IP地址),路由选择是网络层的特殊职责。

\begin{quote}
\textsl{路由选择是指确定数据包的部分或全部路径,而且路由选择的决定是由每个路由器作出的。}
\end{quote}

Internet是网络(LAN、WAN和MAN)的集合,从源到目的地的数据包发送可能是多个发送的组合(例如源到服务器的发送、路由器之间的发送),最后是路由器到目的地的发送。

如果把Internet中的数据包的发送和使用常规邮局服务的信件发送进行比较,可能发现信件的发送者可能使用不同的路径来传递信件,但是最后总能将信息发送到信件的接收者,因此路由选择可能会改变数据包的传递路径,但是源地址和目的地址没有改变。

路由基于目的地址和可用的最佳路径进行选择,而且网络层的路由选择情况是相同的。例如,当一个路由器接收到一个数据包时,它检查路由表并决定该数据包到最终目的地的最佳路线,路由表中提供了下一个路由器的IP地址。当数据包到达下一个路由器时,下一个路由器将会再作出新的决定。

从一个节点到另一个节点传送数据是数据链路层的职责,源计算机使用数据帧封装数据包,在包头增加路由器的数据链路层地址作为目的地址,然后使用广播方式发送数据包。

客户端进程是通过DNS服务器来找到服务器的IP地址的,DNS服务器包含将域名匹配到IP地址的目录,这样当客户端准备和发送信息到DNS服务器时,需要知道其所需要的服务器的实际IP地址,DNS通过逐级查找来获取结果,否则返回错误消息。

\begin{quote}
\textsl{数据传输层负责数据帧的节点到节点的传输,并且使用静态或动态进程来找到下一跳(路由器)的数据链路层地址。}

\textsl{传输层负责客户和服务器之间进程到进程的消息的传输,并建立客户和服务器的传输层的逻辑通信。或者说,传输层负责端到端的数据发送。}
\end{quote}


与IP地址不同,数据链路层的地址不是通用的,每个数据链路层协议可能使用不同的地址格式和大小,而且一个设备可以静态或动态地找到另一个设备地数据链路层地址。

\begin{compactitem}
\item 在静态方法中,设备创建两列的表来存储网络层和数据链路层地址对。
\item 在动态方法中,设备可以广播一个含有下一个设备IP地址的特定数据包,并使用该IP地址询问邻近节点,邻近节点返回它的数据链路层地址。
\end{compactitem}•

数据链路层地址经常被称为物理地址或MAC(介质访问控制)地址,其中在Ethenet(以太网协议)中使用48位地址,通常写成十六进制格式(分成6部分,每部分使用两位十六进制数),例如:
\[00:1E:EC:1E:25:AE\]

有些数据链路层协议在数据链路层使用差错控制和流量控制,方法和传输层相同,不过只是在节点发出点和节点到达点之间实现,因此差错会被检查多次,但是没有一个差错检查覆盖了路由器内部可能发生的差错。

在客户端和服务器端交互时,物理通信是两个物理层之间的(通过许多可能的链路和路由器),传输层可以实现多路复用和解多路复用、拥塞控制、流量控制和差错控制,因此应用层把传输层看作是负责消息传输的代理。

由硬件实现的物理层负责在物理介质上传输二进制流,因此在数据链路层传送的单元是帧,在物理层传送的单元则是二进制位。

物理层不需要地址,帧中的每个位被转化位电磁信号后将通过物理介质(无线或线缆)以广播形式进行传送,因此与发送设备相连的其他设备都能接收到物理层发送的信号。例如,在局域网中,当某台计算机或路由器发送信号时,所有其他的计算机和路由器都将接收到信号。

\begin{compactitem}
\item 数据链路层负责节点之间的帧传送;
\item 物理层负责组成帧的单个二进制位从一个节点到另一个节点的传送。
\end{compactitem}



在实际生产环境中,服务器可能同时运行多个进程(例如FTP服务器进程和HTTP服务器进程),因此除了服务器的IP地址之外,还需要更多因素来实现计算机通信。

当消息到达服务器时,它必须被指向正确的进程,因此用户需要另一个地址(即端口号)来标识服务器进程,而且传输层的职责之一就是多路复用和解多路复用。这里,可以使用类比来说明端口号作为传输层的地址的作用。

假定用户生活在大楼里的不同公寓中,为了向特定的人投递邮件,需要知道大楼的地址和公寓的号码。

\begin{compactitem}
\item IP地址和大楼地址相似;
\item 端口号和公寓号码类似。
\end{compactitem}


端口号是唯一的,而且大多数计算机都包含规定了服务器端口号地址的文件(例如/etc/services),客户端端口号可以由运行客户端进程的计算机临时指定\footnote{Internet对临时端口号的范围进行了限制,以避免破坏通用端口地址范围。}。

\begin{compactitem}
\item 服务器进程使用统一的端口号;
\item 客户端进程使用传输层指定的临时端口号。
\end{compactitem}



在邮件发送和投递的过程中,邮递员无法从每个用户那里收集邮件再把每个邮件分发给目标用户,因此通常就需要由传达室来处理邮件的收集和分发。例如,传达室可以从用户那里统一收集要发出的邮件,然后把它们统一交给邮递员(多路复用),并且还可以把从邮递员手中收集来的邮件逐一分发给用户(解多路复用)。

实际上,传输层为不同的进程做相同的工作,它从进程中收集要发出的信息,并将到达的信息再分发给进程,并且传输层使用端口号(与传达室使用的公寓号码相似)完成多路复用和解多路复用。



另外,物理上传送数据包的下层网络可能发生拥塞,从而导致网络丢失一些数据包,因此传输层还负责实现拥塞控制。例如,有些协议可以为每个进程使用缓冲区,消息在发送前存储在缓冲区中,并且在传输层检测到网络拥塞时暂缓发送,其原理与交通信号灯的效果相似。


传输层在实现流量控制时,发送端的传输层能监控接收端的传输层,并检查接收者发回的确认信息来确定接收者接收到的数据包是否过量。在实现时,接收端确认每个数据包或一组数据包,从而就允许发送者检查接收者接收到的数据包是否过量。

在消息的传输过程中,不可避免地可能发生损坏、丢失、重复或乱序,因此传输层还负责确保消息被目的传输层正确接收。

传输层在提供差错控制时,一般都是在缓冲区(临时存储)中保留消息的副本,直到它从接收者那里接收到数据包无损到达和次序正确的确认信息。如果在预期的时间内没有确认到达或有否定确认(表示数据包被损坏)到达,那么发送者就重新发送数据包。

为了能够检查包的次序,传输层给每个包加上了次序号,给每个确认加上了确认号。


\section{UDP}


TCP/IP协议族中定义了三种传输层协议:UDP、TCP和SCTP。

\begin{compactitem}
\item UDP(用户数据报协议)是传输层协议中最简单的。
\item TCP(传输控制协议)是支持传输层所有职责的协议。
\item SCTP(流控制传输协议)结合了UDP和TCP的优点。
\end{compactitem}




UDP实现多路复用和解多路复用(通过给数据包增加源和目的端口号),而且给包增加校验和来进行差错控制,不过在这种情况下的差错控制只是“是或否”的过程,接收者重新计算校验和,检查在传输中是否有差错发生。

如果接收者确认接收的数据包被损坏,也只是简单地丢弃这个包,并不会通知发送端重新发送,因此UDP不是完美的协议,仅仅在其他职责不是传输层所需要的,或者这些职责已经由应用层完成时才非常有用。

另一方面,UDP的简单使其具有速度快的优点,而且效率高,因此UDP还适合用于及时性比准确性更重要的应用中。例如,在处理视频实时传输时可以使用UDP来实现图像的数据包准时到达,并且图像中小的瞬时错误并不影响用户的观看,因此允许少量的数据包被丢弃或损坏。

与其他协议相比,UDP传送更少的额外信息,而且有些应用程序(例如DNS服务器)使用UDP时可以自己来完成流量控制或差错控制,或者实现快速高效的响应。

UDP被称为无连接协议,不提供属于单个消息的数据包之间的逻辑连接,而且UDP中的每一个包都是一个单独的实体。

类似于常规邮件系统提供的服务,假定用户需要发送一组有次序的包到目的地,UDP(或邮局)不能保证它们按照需要的次序分发,因此用户可以只关心包是否是单独的实体,与其他包没有关系。

TCP(传输控制协议)支持传输层所有职责,但是它没有UDP快速和高效。

TCP使用序号、确认号和校验和,而且TCP在发送端使用缓冲区,以及支持多路复用、解多路复用、流量控制、拥塞控制和差错控制等。

与UDP不同,TCP在两个传输层之间提供逻辑连接,因此是面向连接的协议。

实际上,一个在源端,一个在目的端,序号的使用维持了连接。

\begin{compactitem}
\item 接收端的数据包到达的顺序错了或丢失了时,将被重新发送。
\item 接收端的传输层不把次序错的数据包发送给应用进程。
\item 接收端保留消息中的所有数据包,直到它们以正确的次序被接收。
\end{compactitem}

TCP是数据通信中完美的传输层协议,但是不适合音频和视频的实时传输。例如,如果数据包丢失,TCP需要重新发送,这样就破坏了数据包的同步。

SCTP结合了UDP和TCP的优点来支持某些预期的Internet服务(例如因特网电话和视频流)。

\begin{compactitem}
\item SCTP适合用于音频和视频的实时传输;
\item SCTP提供差错控制和流量控制。
\end{compactitem}








\chapter{Email}



Email(电子邮件)是两个实体间的消息交换,不能被客户端进程和服务器端进程支持。



\chapter{Telnet}




















































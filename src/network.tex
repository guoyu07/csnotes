\part{Network}

\chapter{Overview}


计算机网络是硬件和软件的组合,其中硬件把信号从网络中的一个节点传送到另一个节点,而软件用于使用户可以使用网络服务。

网络的性能可以用多种方式来衡量,例如传输时间、响应时间等。

\begin{compactitem}
\item 传输时间是信息从一个设备传输到另一个设备所需的时间总量。
\item 响应时间是查询和响应之间的时间间隔。
\end{compactitem}

网络的性能依赖于许多因素,例如用户数、传输介质类型、硬件的连接能力和软件的效率等。

除了发送的准确性之外,还可以从发生故障的频率、从故障中恢复的时间、灾难时网络的健壮性等因素来衡量网络的可靠性。

网络的安全问题包括数据保护、防范非授权访问、破坏和修改,实现从数据破坏和数据丢失中恢复的策略和程序等。

\section{Connection}

链路是数据从一个设备传输到另一个设备的通信信道,可能的连接类型包括点对点和多点。

\begin{compactitem}
\item 点对点连接提供了两个设备间的专用链路,而且链路的整个容量为两个设备的传输所拥有。
\item 多点连接(或多站连接)是两个以上的指定设备共享一个链路,因此信道的容量被共享。
\end{compactitem}

\section{拓扑}

网络的拓扑是所有链路和设备(通常为节点)间关系的几何表示,包括网状型、星型、总线型和环型。

实际上,网络的物理拓扑都是两个或多个设备连接到一个链路,一个或多个链路形成拓扑。

\begin{compactitem}
\item 在网状拓扑中,每个设备都有专用的点对点链路与其他每个设备相连。
\item 在星型拓扑中,每个设备都有专用的点对点链路,只与集线器(Hub)相连。
\item 在总线拓扑中,所有的设备都连接到总线,而且每个节点使用分支线和连接器与总线相连。
\item 在环形拓扑中,每个设备都有专用的点对点链路,并且只与两边的设备相连,信号只是以一个方向沿着环从一个设备传输到另一个设备,直到到达目的地。
\end{compactitem}•

在环形拓扑中,环中的每个设备连接一个中继器(Router),这样当一个设备收到要发送到另一个设备的信号时,中继器可以重新生成二进制位并传输它们。

星型拓扑比网状拓扑更便宜,而且具有网状拓扑的大多数优点,只是其整个拓扑依赖单个点(即集线器)。

\section{Router}



当两个或多个网络通过路由器连接在一起时,它们就变成了互联网(internet),其中最著名的互联网就是因特网(Internet)。

路由器(Router)是发送数据包(消息),并使其在互联网中传输的连接设备。

\begin{compactitem}
\item 网络本身可以是一组连接在一起的通信设备(例如计算机和打印机)。
\item 互联网本身是能够通过路由器进行互相通信的两个或多个网络。
\end{compactitem}

\chapter{Protocol}


\section{TCP}


使用计算机解决问题的基础工作是由硬件完成的,用户通过软件可以控制问题求解过程,硬件工作的细节问题由软件层处理。

计算机网络提供的服务也可以类比于解决问题,例如发送电子邮件的任务可以被分解为更小的子任务。

实际上,计算机网络的每层完成一个任务,而且每层都使用更低层的服务。

在计算机网络的最底层,信号或信号组被从源计算机传送到目的计算机,这也是网络协议的最底层。

计算机网络协议允许使用不同技术的LAN和WAN互相连接到一起,从而可以从一点向另一点传送信息。

原始的TCP/IP协议族被定义为4层,分别是主机到网络层(或链路层)、互联网层(网络层)、传输层和应用层,现在已经发展为5层。

\begin{table}[htbp]
\centering
\caption{TCP/IP协议族}
\begin{tabular}{|l|l|l|l|l|}
\hline
	& 应用层 & 5 & 应用层 & \\ \cline{2-4}
	&传输层 & 4 & 传输层 & \\ \cline{2-4}
TCP/IP&网络层 & 3 & 互联网层 & 原始的TCP/IP\\ \cline{2-4}
	&数据链路层 & 2 & 主机到网络层 & \\ \cline{2-4}
	&物理层 & 1 & & \\
\hline
\end{tabular}
\end{table}

在网络协议分层结构中,每一层调用其直接下层的服务,路由器只使用TCP/IP协议族的前三层。

应用层允许用户(人或者软件)访问网络,并提供对电子邮件、远程文件访问和传输、浏览Internet等服务的支持。

应用层负责向用户提供服务,而且应用层是唯一一个大多数Internet用户能够直观感受的层。例如,客户/服务器体系结构和对等体系结构都可以用于允许不同计算机上的应用程序互相通信,它们都需要应用层的支持。

在客户/服务器体系结构中,每个应用由两个分开但相关的程序(客户端程序和服务器端程序)组成。

\begin{compactitem}
\item 客户端程序只在需要时运行。
\item 服务器端程序需要一直运行。
\end{compactitem}

客户端程序和服务器端程序之间的通信称为进程到进程的通信,其中服务器端进程一直运行,并等待接收客户端进程的请求。

用户运行休眠的客户端程序就可以将其转变为客户端进程,使得客户端进程请求服务,而且将被服务器端进程响应。

一般情况下,服务器端进程可以响应多个客户端进程的请求并返回数据。

\section{IP}


当客户需要向服务器发送请求时,首先需要服务器应用层的地址,而且服务器应用层地址不能用来发送消息的,它只是帮助客户找到服务器的实际地址。

在应用层,客户端位置不需要进行标识,不同的服务器有不同的应用层地址。举例来说,在只知道一个人的姓名时无法向其发送信件,邮局无法仅凭姓名投递信件,还需要收信人的实际地址。

应用层地址可以帮助客户端找到服务器的实际地址,也就是服务器的IP地址。

实际上,客户端进程是通过DNS服务器来找到服务器的IP地址的,DNS服务器包含将域名匹配到IP地址的目录。

客户端准备和发送信息到DNS服务器,询问其所需要的服务器的实际IP地址,DNS逐级查找并获取结果,否则返回错误消息。

传输层负责客户和服务器之间进程到进程的消息的传输,并建立客户和服务器的传输层的逻辑通信,因此应用层可以把传输层看作是负责消息传输的代理。
















































